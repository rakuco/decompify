\documentclass{article}
\usepackage{graphicx}
\usepackage[normalem]{ulem} % needed by strike
\usepackage[urlcolor=blue,colorlinks=true]{hyperref}
\usepackage[utf8]{inputenc}  % char encoding

\title{\textbf{MC404 2008s2 - Projeto 2}}
\author{Grupo 30\\
071294 - Jorge Augusto Hongo\\
072201 - Raphael Kubo da Costa}
\begin{document}
\maketitle

\section{Introdução}
Este relatório detalha algumas escolhas de design e implementação e problemas enfrentados durante a elaboração do segundo projeto de MC404 do segundo semestre de 2008.

Este projeto foi escrito e testado diretamente no \textbf{Linux}, utilizando registradores de 32 bits e as chamadas de sistema nativas do sistema operacional. Os comentários no código estão em inglês por simples convenção e costume.


\section{Estrutura do código}
As tarefas de modularizar o código e dividi-lo em pequenas unidades funcionais provou-se bem mais difícil que no primeiro projeto.

Ao contrário do projeto anterior, a maior parte das funções e macros encontra-se no arquivo \textit{main.asm}. Arquivos auxiliares, como \textit{util.h} ou \textit{string.asm} contêm funções e macros agrupadas por semelhança de funcionalidades.

No final, o código foi dividido nos seguintes arquivos:

\begin{description}
\item[\textbf{disasm.asm}]
  Subrotinas, definições e macros relaciondas à manipulação do arquivo de disassembly.
\item[\textbf{file.asm}]
  Subrotinas relaciondas à manipulação de arquivos em geral.
\item[\textbf{instr.h}]
  Definições estáticas da estrutura de cada opcode e listagem de registradores e modos de endereçamento.
\item[\textbf{main.asm}]
  Subrotina principal, que controla e chama as demais. Possui o \textit{core} do programa.
\item[\textbf{math.h}]
  Macros relacionadas a operações matemáticas.
\item[\textbf{opcodes2asm.py}]
  Script para geração do arquivo \textit{opcodes.h} com base em \textit{opcodes.dat}.
\item[\textbf{opcodes.dat}]
  Arquivo de texto usado como base para a geração da listagem de opcodes em Assembly.
\item[\textbf{opcodes.h}]
  Gerado por \textit{opcodes2asm.py}. Contém a lista de opcodes, seus atributos e os grupos usados pelo programa.
\item[\textbf{string.\{asm,h\}}]
  Subrotinas, definições e macros relaciondas à manipulação de strings de maneira simplificada.
\item[\textbf{syscalls.h}]
  Macros e definições para as chamadas de sistema do Linux (abertura e fechamento de arquivos, seeks e afins).
\end{description}

Em geral, em todas as macros e subrotinas, tentou-se preservar o conteúdo dos registradores passados, exceto em casos em que era necessário retornar algum valor no registrador \textit{EAX}.


\section{Funcionamento do programa}
A estrutura do programa tem como base o Disasm, \textit{disassembler} escrito por Marius Gedminas, disponível em \textbf{http://gemin.as/disasm}. Seu programa, com versões em Pascal e Assembly, desmonta arquivos .COM para o formato de Assembly utilizado pelo TASM.

Do Disasm foi utilizada em especial a idéia da criação de camadas de indireção para referenciar as propriedades e informações de cada instrução do 8086. Para alguns trechos e principalmente para consulta foi utilizada a biblioteca \textbf{libASM}, de Christian Fowelin (\textit{http://www.fowelin.de/christian/computer/libasm}).

\subsection{Algoritmo e estrutura utilizados}
Basicamente, o programa realiza uma busca na tabela de opcodes de \textit{opcodes.h}: cada byte lido possui uma entrada correspondente na tabela com informações como o tamanho dos operandos, seus tipos e se pertence a algum grupo de opcodes.

Todas as informações da entrada são processadas e guardadas em variáveis apropriadas. Essas informações são então utilizadas para escrever no arquivo de disassembly a instrução correspondente e seus operandos devidamente formatados.

Nomes de registradores são definidos em \textit{instr.h}, e há estruturas como \texttt{ARRAY\_16BITREGS} ou \texttt{ARRAY\_CONSTARGS} que possuem o endereço de memória das instruções ou strings dos registradores desejados. Por sua vez, os opcodes disponíveis na tabela costumam conter a posição de determinado item dessas estruturas. Para a escrita no arquivo final, é necessário passar por várias camadas de indireção.

Grande parte da complexidade do projeto advém da estrutura dos opcodes do 8086: como este programa decodifica uma instrução por vez, sem informações sobre as anteriores ou as próximas, é necessário realizar muitas checagens sobre os dados disponíveis em tabela.


\section{Conclusão}
\subsection{Complexidade de código}
Devido à maneira como as instruções são representadas no 8086 e como os opcodes são estruturados, muitas macros ficaram bastante grandes e com uma quantidade maior que a desejável de checagens de parâmetros. Em uma linguagem de mais alto nível, essa quantidade de código e checagem ficaria grande, e em Assembly é possível entender o tamanho a que tenha chegado.

Fez-se bastante uso do preprocessador do nasm, tanto para definir macros simples quanto para realizar checagens de parâmetros em macros maiores ou mais complexas, como \texttt{StoreData}. Na medida do possível, tentou-se criar macros e procedimentos que facilitassem o desenvolvimento. Como exemplo, foram criados \textit{wrappers} para a macro \texttt{sys\_write}, por sua vez já uma abstração sobre a chamada de sistema \texttt{write}: \texttt{print\_string}, \texttt{printnl} e \texttt{write\_string} tornam a programação um pouco menos dolorosa. A macro \texttt{exec}, por sua vez, torna a chamada de uma função que recebe argumentos na pilha mais parecida com uma chamada comum de linguagem de alto nível.

O projeto e o algoritmo (não muito eficiente) são bastante simples. Em C ou em alguma linguagem qualquer de mais alto nível, seria possível implementar o projeto em algumas horas e com um número muito menor de linhas de código.

\subsection{Problemas de remontagem com o nasm}
Não foi possível realizar testes satisfatórios com o nasm após a desmontagem com o programa. Algumas instruções, como \texttt{mov bx, bx}, precisam ser representadas como um opcode mas o nasm utiliza outro equivalente. Por isso, programas como o quinto arquivo de teste (\textit{05.COM}) não puderam ser remontados com sucesso.

\subsection{Bugs}
Durante os testes, percebeu-se que, caso o último (ou os últimos) bytes correspondam a instruções e não sejam completados (isso ocorre quando o final do arquivo não corresponde a código, mas a dados, e o último byte é algo como 0x00 (ADD), que necessitaria de outros bytes como argumentos). Mesmo depois do último byte, o programa procura mais bytes e os escreve (0x00). Na remontagem, não há grandes problemas, uma vez que os novos bytes não são referenciados.

Embora não se trate exatamente de um bug, não foi possível implementar a separação clara entre a seção de código e a seção de dados, como ocorre no \textbf{Disasm}.

\subsection{Código aberto}
Da mesma maneira que o projeto anterior, que encontra-se hospedado no \textbf{Launchpad} com seu código aberto, este projeto teve suas versões gerenciadas pelo \textbf{git} e está hospedado no github em \texttt{`http://github.com/rakuco/decompify`}.

O programa foi batizado como \textbf{decompify} e foi liberado sob a licença GPLv3.

\end{document}
